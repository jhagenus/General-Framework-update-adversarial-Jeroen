% Setup figures
\usepackage[caption=false,font=normalsize,labelfont=sf,textfont=sf]{subfig}
\usepackage{graphicx}
\usepackage{standalone}
\usepackage{tikz}
\usetikzlibrary{shapes,backgrounds,shapes.geometric}
\usepackage{circuitikz}
\tikzset{font={\fontsize{9pt}{9}\selectfont}}
\usepackage{pgfplots}
\usepgfplotslibrary{fillbetween,colormaps,groupplots}
\pgfplotsset{compat=newest}


% Setup tables
\usepackage{multirow}
\usepackage{tabularx}
\usepackage{makecell}
\usepackage{booktabs}
\setlength{\extrarowheight}{1mm}
\newcolumntype{Y}{>{\raggedleft\arraybackslash}X}
\newcolumntype{Z}{>{\centering\arraybackslash}X}
\newcolumntype{L}{>{\hsize=0.61\hsize}X}
\newcolumntype{C}{>{\hsize=0.61\hsize}Z}
\newcolumntype{R}{>{\hsize=0.61\hsize}Y}
\newcolumntype{K}{>{\hsize=0.33333\hsize}X}
\newcolumntype{M}{>{\hsize=0.33333\hsize}Z}
\newcolumntype{N}{>{\hsize=0.33333\hsize}Y}


% Setup equations
\usepackage{siunitx}
\usepackage{amsmath}
\usepackage{amsfonts}
\usepackage{scalerel}
\usepackage{amssymb}
\usepackage{bm}


% Setup algorithms
\usepackage{algorithm}
\usepackage{algorithmicx}


% Define additional symbols
\DeclareRobustCommand{\bbone}{\text{\usefont{U}{bbold}{m}{n}1}}
\DeclareMathOperator{\EX}{\mathbb{E}}% expected value
\DeclareMathOperator*{\Aop}{\scalerel*{\mathbb{A}}{\sum}}
\DeclareMathOperator{\sign}{\text{sgn}}


% Setup design
\usepackage[utf8]{inputenc}
\usepackage{enumitem}

% Setup title
\date{\today}


% Setup Pdf metadata
\usepackage[
	pdfauthor={Julian Schumann},
	pdfcreator={pdfLatex},
	pdfsubject={First project for PhD},
    colorlinks={true},
    linkcolor={blue},
    citecolor={blue},
    urlcolor={red},
    hyperindex={true}
]{hyperref}
\makeatletter
\hypersetup{pdftitle=\@title}
\makeatother
\setcounter{secnumdepth}{6}


% Define subscript location
\newcommand\newss{\setbox0=\hbox{\(\)}%
\fontdimen16\textfont2=1ex
\fontdimen17\textfont2=1ex
}


% Setup for aligned commotions model name
\usepackage{mathtools}
\DeclareMathOperator{\IoA}{\mathrlap{I}\hphantom{A}}
\DeclareMathOperator{\IoN}{\mathrlap{I}\hphantom{N}}
\DeclareMathOperator{\JoA}{\mathrlap{J}\hphantom{A}}